\documentclass[a4paper,10pt]{article}

\usepackage{stackrel, amsmath,amsthm,amsfonts,amssymb, stmaryrd}
%\usepackage{amsthm}
%\usepackage{mathrsfs}
%\usepackage{amsfont}
\usepackage{dsfont}\let\mathbb\mathds
\usepackage{yhmath}
\usepackage{listings}
\lstset{breaklines=true}
\usepackage{xr}


%PGF/TikZ
\usepackage{pgf,pgfplots, tikz}
\usetikzlibrary{arrows}

\usepackage{hyperref}
\hypersetup{
    colorlinks,
    citecolor=black,
    filecolor=black,
    linkcolor=black,
    urlcolor=black
}

\usepackage{float}
\usepackage[utf8]{inputenc} 
%\usepackage[T1]{fontenc} 
\usepackage[english]{babel}
\usepackage[all]{xy}
\usepackage{graphicx} 
\usepackage{geometry}
\geometry{hmargin=2.5cm,vmargin=2cm}

\usepackage{fancyvrb}

\usepackage{setspace}
\onehalfspacing 

\usepackage{mathtools}

\usepackage[ruled,linesnumbered]{algorithm2e}

\usepackage[
n, % or lambda advantage , operators , sets,
adversary , landau , probability , notions ,
logic ,
ff, mm,
primitives , events , complexity , asymptotics , keys
]{cryptocode}

\theoremstyle{definition}
\newtheorem{definition}{Definition}[section]

\theoremstyle{plain}
\newtheorem{proposition}[definition]{Proposition}
\newtheorem{lemma}[definition]{Lemma}
\newtheorem{corollary}[definition]{Corollary}
\newtheorem{theorem}[definition]{Theorem}

\theoremstyle{definition}
\newtheorem{remark}[definition]{Remark}
\newtheorem{example}[definition]{Example}
\newtheorem{assumption}[definition]{Assumption}
\newtheorem{problem}[definition]{Problem}

 %macros
\newcommand{\N}{\mathbb{N}}
\newcommand{\Z}{\mathbb{Z}}
\newcommand{\Q}{\mathbb{Q}}
\newcommand{\R}{\mathbb{R}}
\newcommand{\C}{\mathbb{C}}
\newcommand{\K}{\mathbb{K}}
\newcommand{\F}{\mathbb{F}}
\newcommand{\Hc}{\mathbb{H}}
\newcommand{\Lc}{\mathbb{L}}
\newcommand{\M}{\mathbb{M}}
\newcommand{\Pc}{\mathbb{P}}
\newcommand{\A}{\mathbb{A}}
\newcommand{\B}{\mathbb{B}}
\newcommand{\E}{\mathbb{E}}
\newcommand{\V}{\mathbb{V}}
\newcommand{\m}[1]{\mathcal{#1}}
\newcommand{\mA}{\mathcal{A}}
\newcommand{\mB}{\mathcal{B}}
\newcommand{\mC}{\mathcal{C}}
\newcommand{\mP}{\mathcal{P}}
\newcommand{\mK}{\mathcal{K}}
\newcommand{\mL}{\mathcal{L}}
\newcommand{\mE}{\mathcal{E}}
\newcommand{\mD}{\mathcal{D}}
\newcommand{\mF}{\mathcal{F}}
\newcommand{\mO}{\mathcal{O}}
\newcommand{\im}{\mbox{im}}
\renewcommand{\ker}{\mbox{ker}}
\renewcommand{\dim}{\mbox{dim}}
\renewcommand{\deg}{\mbox{deg}}
\renewcommand{\i}[2]{\left\llbracket #1~;~#2\right\rrbracket}
\renewcommand{\(}{\left(}
\renewcommand{\)}{\right)}
\renewcommand{\P}{\mathbb{P}}
%commandes de Quentin
%\newcommand{\F}[1]{\mathbb{F}_{#1}}
\newcommand{\Fp}[1]{\mathbb{F}_{p^{#1}}}
\newcommand{\db}[1]{\mathbb{#1}}
\newcommand{\id}{\mbox{id}}
\newcommand{\mf}[1]{\mathfrak{#1}}
\newcommand{\mfm}{\mathfrak{m}}
\newcommand{\mfn}{\mathfrak{n}}
\newcommand{\mfp}{\mathfrak{p}}
\newcommand{\mfq}{\mathfrak{q}}
\newcommand{\Spec}{\mbox{Spec}}
\newcommand{\Proj}{\mbox{Proj}}
\newcommand{\Hom}{\mbox{Hom}}
\newcommand{\End}{\mbox{End}}
\newcommand{\leftmapsto}{\leftarrow\!\shortmid}
\newcommand{\longleftmapsto}{\longleftarrow\!\shortmid}
\renewcommand{\vec}[1]{\mathbf{#1}}

\begin{document}

\begin{center}

{\LARGE {\bfseries A Lattice Based Cryptanalysis of Oriented Supersingular Isogeny Diffie-Hellman}}

\vspace{1cm}

{\Large Pierrick Dartois\footnote{IBM Research GmbH - Zurich, Universit\'{e} de Rennes 1, Corps des Mines} and Luca De Feo\footnote{IBM Research GmbH - Zurich}}

\vspace{1cm}

\end{center}

\textbf{Abstract:} We present an attack on the Oriented Supersingular Isogeny Diffie-Hellman key exchange due to Leonardo Col\`{o} and David Kohel  \cite{OSIDH} based on the ideal class group action of an order of a quadratic imaginary field $\mO_n\subset K$ with prime power conductor $\ell^n$ on oriented descending $\ell$-isogeny chains. This attack is based on an idea due to Hiroshi Onuki \cite{Onuki}.  As the conceptors of OSIDH noticed themselves \cite[§ 5.3]{OSIDH}, to break the key encapsulation mechanism, it suffices to be able to recover the whole descending $\ell$-isogeny chain given the first and the deepest elliptic curves of the chain $E_0$, $E_n$ and some additional data on the class group action on $E_n$. Onuki noted that one can recover the whole isogeny chain with the knowledge of $K$-oridented endomorphisms at each level of the chain.  However, Onuki's approach to this problem based on meet-in-the-middle and an exhaustive search of powersmooth elements of $\mO_n$ can be dramatically improved with an approach based on lattices.

\section{Introduction to the problem}

Let $K$ be a quadratic imaginary number field with trivial ideal class group $\mbox{Cl}(\mO_K)=\{1\}$, $\ell$ a small prime number, $n\in\N^*$, $p$ be a big prime number that does not split in $K$ and $\mf{q}_1, \cdots, \mf{q}_t\subseteq \mO_K$ a set of prime ideals lying above distinct splitting primes $q_1, \cdots, q_t$ distinct from $\ell$. For all $i\in\N$, let $\mO_i:=\Z+\ell^i\mO_K$. We assume that the prime ideals $\mf{q}_j$ (when intersected with $\mO_n$) generate $\mbox{Cl}(\mO_n)$. 

A descending $\ell$-isogeny chain $(E_i, \iota_i)_{0\leq i\leq n}$ is a sequence of descending $\ell$-isogenies:
\[(E_0, \iota_0)\overset{\varphi_0}{\longrightarrow}(E_1,\iota_1)\cdots(E_{n-1},\iota_{n-1})\overset{\varphi_{n-1}}{\longrightarrow}(E_n,\iota_n)\]
such that $(E_i, \iota_i)$ is an $\mO_i$-orientation for all $i\in\i{0}{n}$. The OSIDH cryptosystem is based on a restricted effective group action of $\mbox{Cl}(\mO_n)$ on descending $\ell$-isogeny chains represented as sequences of $j$-invariants.  We know how to compute efficiently the action of ideals of the form $\mf{a}=\prod_{j=1}^t\mf{q}_j^{e_j}$ where the exponents $e_j$ all lye in a small integer range $\i{-r}{r}$. 

To break the OSIDH Diffie-Hellman key exchange, one has to solve the following problem:

\begin{problem}\label{problem 1}
Let $(E_i,\iota_i)_{0\leq i\leq n}$ be a descending $\ell$-isogeny chain, $\mf{a}:=\prod_{j=1}^t\mf{q}_j^{e_j}$ with $e_1,\cdots, e_t\in\i{-r}{r}$ a secret ideal and:
\[(F_i,\iota'_i)_{0\leq i\leq n}:=\mf{a}\cdot (E_i,\iota_i)_{0\leq i\leq n}\]
be the result of the class group action of $\mf{a}$ on $(E_i,\iota_i)_{0\leq i\leq n}$. Then, with the knowledge of the chain $(E_i,\iota_i)_{0\leq i\leq n}$, the ending elliptic curve $F_n$, and the horizontal chains:
\[\mf{q}_j^{-r}\cdot F_n:=F_n/F_n[\overline{\mf{q}}_j^r]\longrightarrow \cdots \longrightarrow F_n\longrightarrow \cdots\longrightarrow \mf{q}_j^{r}\cdot F_n:=F_n/F_n[\mf{q}_j^r]\]
for all $j\in\i{1}{n}$, the problem is to recover the ideal class of $[\mf{a}]$ and more precisely, exponents $e'_1, \cdots, e'_t\in\i{-r}{r}$ such that $[\mf{a}]=\prod_{j=1}^t [\mf{q}_j]^{e'_j}$. \footnote{These exponents do not have to be the exact exponents $e_1, \cdots, e_t$ used to generate $\mf{a}$. Only the class of $\mf{a}$ matters.}
\end{problem}

As Leonardo Col\`{o} and David Kohel noted \cite[§ 5.3]{OSIDH}, there is a reduction of problem \ref{problem 1} to the following:

\begin{problem}\label{problem 2}
Let $(E_i,\iota_i)_{0\leq i\leq n}$ be a descending $\ell$-isogeny chain. Then, with the knowledge of the first elliptic curve $E_0$, the ending elliptic curve $E_n$, and the chains:
\[\mf{q}_j^{-r}\cdot E_n:=E_n/E_n[\overline{\mf{q}}_j^r]\longrightarrow \cdots \longrightarrow E_n\longrightarrow \cdots\longrightarrow \mf{q}_j^{r}\cdot E_n:=E_n/E_n[\mf{q}_j^r]\]
for all $j\in\i{1}{n}$, the problem is to recover the chain $(E_i,\iota_i)_{0\leq i\leq n}$.
\end{problem}

\section{Onuki's attack}

We present an attack due to Hiroshi Onuki \cite[§ 6.3]{Onuki} recovering the descending $\ell$-isogeny chain:
\[(E_0, \iota_0)\overset{\varphi_0}{\longrightarrow}(E_1,\iota_1)\cdots(E_{n-1},\iota_{n-1})\overset{\varphi_{n-1}}{\longrightarrow}(E_n,\iota_n)\] 
given $E_0$ and $E_n$ together with the horizontal chains:
\[\mf{q}_j^{-r}\cdot E_n\longrightarrow \cdots \longrightarrow E_n\longrightarrow \cdots\longrightarrow \mf{q}_j^{r}\cdot E_n \quad (j\in\i{1}{t})\quad (\star)\]

Assume that the attacker knows an endomorphism $\iota_n(\beta)$ for a known value $\beta\in\mO_n\setminus \mO_{n+1}$ that we write $\beta:=a+b\ell^n\theta$, where $\theta$ is a generator of $\mO_K$ and $a,b\in\Z$, with $b\wedge \ell=1$. Since $\iota_n(a)=[a]$ is easy to compute, we can assume that $a=0$ i.e. that $\beta=b\ell^n\theta$. We assume that $\iota_n(\beta)$ can be efficiently evaluated on $\ell$-torsion points (i.e. in polynomial time in $n$). Then the attacker can compute the subgroup $G:=\ker(\iota_n(\beta))\cap  E_n[\ell]$ in polynomial time in $n$. 

\begin{lemma}
$G=\ker(\widehat{\varphi}_{n-1})$.
\end{lemma}

\begin{proof}
We have:
\[\iota_n(\beta)=\iota_n(b\ell^n\theta)=[\ell]\iota_n(b\ell^{n-1}\theta)=\varphi_{n-1}\iota_{n-1}(b\ell^{n-1}\theta)\widehat{\varphi}_{n-1}\]
and $b\ell^{n-1}\theta\in\mO_{n-1}$, so that $\iota_{n-1}(b\ell^{n-1}\theta)\in\End(E_{n-1})$, and consequently, $\ker(\widehat{\varphi}_{n-1})\subseteq\ker(\iota_n(\beta))$. Since $\deg(\varphi_{n-1})=\ell$, we have also $\ker(\widehat{\varphi}_{n-1})\subseteq E_{n}[\ell]$ so that $\ker(\widehat{\varphi}_{n-1})\subseteq G$.  So $G$ is either cyclic of order $\ell$ and equal to $\ker(\widehat{\varphi}_{n-1})$ or of order $\ell^2$ and equal to the entire $\ell$-torsion subgroup $E_n[\ell]$. If the latter holds, $\iota_n(\beta)$ factors through $[\ell]$ by \cite[corollary III.4.11]{Silverman1} and $\beta/\ell=b\ell^{n-1}\theta\in\mO_n$, so $\ell|b$. Contradiction. Hence, $G=\ker(\widehat{\varphi}_{n-1})$.
\end{proof}

Hence, we can compute $\widehat{\varphi}_{n-1}$ using V\'{e}lu's formulas \cite{Velu} in $O(\ell)$ operations over the field of definition of $E_n[\ell]$. With this information, we can recover $\varphi_{n-1}$ easily, by evaluating $\widehat{\varphi}_{n-1}$ on $E_{n-1}[\ell]$, since $\ker(\varphi_{n-1})=\widehat{\varphi}_{n-1}(E_{n-1}[\ell])$, and using V\'{e}lu's formulas again.

We conclude that the attacker can compute the chain of isogenies $(\varphi_i : E_i\longrightarrow E_{i+1})_{0\leq i\leq n-1}$, if at each index $i\in\i{1}{n}$,  they have access to an oracle providing $\iota_i(\beta_i)$ for $\beta_i\in\mO_i\setminus\mO_{i+1}$, when $E_{i}$ is given. 

For $i=n$,Onuki's idea to provide this oracle is the following.  We look for $\beta\in\mO_n\setminus\mO_{n+1}$ such that $\beta\mO_K:=\mf{b}\mf{c}$ with $\mf{b}:=\prod_{j=1}^t\mf{q}_j^{e_j}$ and $e_1, \cdots, e_t\in\i{-r}{r}$ big enough to make the norm of $\mf{c}$ as small as possible. Then, we compute the isogeny associated to $\mf{b}$:
\[\varphi_{\mf{b}}: E_n\longrightarrow \mf{b}\cdot E_n\]
with the help of the chains $(\star)$ as explained in \cite[§ 5.2]{OSIDH}. To compute the isogeny associated to $\mf{c}$:
\[\varphi_{\mf{c}}: \mf{b}\cdot E_n\longrightarrow E_n\]
Onuki recommends a meet-in-the-middle approach (without any indication to make it effective). Combining the exhaustive search of $\beta\in \mO_n\setminus\mO_{n+1}$ with a big factor $\mf{b}$, i.e. whose norm is divisible by many powers of the $q_j$ and the meet-in-the-middle approach to find $\mf{c}$, the attack is exponential, so we need a way to make the oracle more efficient.


\section{An improvement based on short vector compuation in a euclidean lattice}

Given $E_0$ and $E_n$ and the chains $(\star)$, we recover the whole chain $(E_i,\iota_i)_{0\leq i\leq n}$ with an oracle returning an endomorphism $\iota_i(\beta_i)$ with $\beta_i\in\mO_i\setminus\mO_{i+1}$ when given $E_i$ for $i\in\i{1}{n}$. However, the oracle is different here.  Instead of searching for $\beta\in\mO_n\setminus\mO_{n+1}$ (for $i=n$) with smoothness conditions on its norm coupled with a meet-in-the-middle attack, we directly look for $\beta\in\mO_n\setminus\mO_{n+1}$ as a product of the $\mf{q}_j$ with exponents in $\i{-2r}{2r}$ by solving the equation:
\[\prod_{j=1}^t[\mf{q}_j]^{e_j}=[1]\]
in $\mbox{Cl}(\mO_n)$, with $e_1,\cdots, e_t\in\i{-2r}{2r}$ non-trivial.  Then, we write $e_j:=e'_j+e''_j$ with $e'_j,e''_j\in\i{-r}{r}$ for all $j\in\i{1}{t}$ and compute the isogenies:
\[\varphi : E_n \longrightarrow \prod_{j=1}^t[\mf{q}_j]^{e'_j}\cdot E_n \qquad \mbox{and} \qquad \psi : E_n  \longrightarrow  \prod_{j=1}^t[\mf{q}_j]^{-e''_j}\cdot E_n=\prod_{j=1}^t[\mf{q}_j]^{e'_j}\cdot E_n\]
and finally compute $\iota_n(\beta)=\hat{\psi}\circ\varphi$ with $\beta\mO_K= \prod_{j=1}^t\mf{q}_j^{e_j}$.

Hence, our problem is solved if we can compute the lattice:
\[L:=\left\{(e_1,\cdots,e_t)\in\Z^t\middle| \  \prod_{j=1}^t[\mf{q}_j]^{e_j}=[1]\right\}\]
and find a short vector in $L$ whose norm $\ell_\infty$ is $\leq 2r$ to make sure that we can compute $\iota_n(\beta)$ by the meet-in-the-middle argument presented above.  

\subsection{On the existence of short vectors in $L$}

Assuming that the $[\mf{q}_j]$ generate $\mbox{Cl}(\mO_n)$, $L$ has covolume $h(\mO_n)$. Assuming that $L$ behaves as a uniformly random lattice among lattices of covolume $h(\mO_n)$, we may assume that:
\[\lambda_1^{(\infty)}(L)\leq \(1+\frac{\log\log(t)}{t}\)\frac{h(\mO_n)^{\frac{1}{t}}}{2}\]
This result can be proved with the techniques of Aono, Espitau and Nguyen \cite[§ 3]{Aono}.

To ensure that the key space is sufficiently large to cover $\mbox{Cl}(\mO_n)$, we require the surjectivity of the map:
\[f:(e_1,\cdots, e_t)\in\i{-r}{r}^t\longmapsto \prod_{j=1}^t[\mf{q}_j]^{e_j}\in G\]
It follows that $h(\mO_n)\leq (2r+1)^t$, so that:
\[2r\geq h(\mO_n)^{\frac{1}{t}}-1>\(1+\frac{\log\log(t)}{t}\)\frac{h(\mO_n)^{\frac{1}{t}}}{2}\geq \lambda_1^{(\infty)}(L)\]
Hence, we are confident that a sufficiently short vector exists in this lattice and therefore, that the attack is theoretically possible in general.

\subsection{Computing a basis of $L$}

We can find a basis of $L$ in polynomial time in $n$ and $t$ with the following method. First, we extract from the generating set $\{[\mf{q}_1],\cdots, [\mf{q}_t]\}$ a basis $\m{B}:=(g_1, \cdots, g_r)$ of $\mbox{Cl}(\mO_n)$. In general, this is very simple because $\mbox{Cl}(\mO_n)$ is cyclic generated by one of the $[\mf{q}_j]$. When $\mbox{Cl}(\mO_n)$ is not cyclic, this can be done with Sutherland's algorithm \cite[algorithm 4]{Sutherland2010} for each Sylow subgroup of $\mbox{Cl}(\mO_n)$. The computation remains polynomial because $\mbox{Cl}(\mO_n)$ has a big $\ell$-Sylow subgroup of rank $\leq 4$ (as a consequence of \cite[theorem 4.2.10]{Cohen2}) and that other Sylow subgroups are very small.

With this basis, we can compute the discrete logarithms of the $[\mf{q}_j]$ in $\m{B}$, i.e.obtain $x_{1,j}, \cdots, x_{r, j}\in\Z$ such that:
\[[\mf{q}_j]=\prod_{i=1}^r g_i^{x_{i,j}}\]
for all $j\in\i{1}{t}$. Again, this can be done in polynomial time (using \cite[algorithm 2]{Sutherland2010} for instance) because the prime factors of $h(\mO_n)$ and number of elements $r$ of $\m{B}$ is small. 

Then, we get that the lattice $L$ is given by:
\[L=\left\{(e_1, \cdots, e_t)\in\Z^t\middle|\ \forall j\in\i{1}{r}, \quad \sum_{j=1}^t x_{i,j}e_j\equiv 0 \ [|g_i|]\right\}\]
$|g_i|$ being the order of $g_i$ for all $i\in\i{1}{r}$.  

It is easier to compute the dual lattice of $L$:
\[L^*=\Z^t+\sum_{i=1}^r|g_i|^{-1}\Z\cdot x_i\]
where $x_i:=(x_{i,j})_{1\leq j\leq t}$ for all $i\in\i{1}{r}$.  Once we have obtained a basis $C$ of $L^*$ (for instance by computing the Hermite normal form of $\(m/|g_1|x_1^T|\cdots|m/|g_r|x_r^T|mI_t\)\in M_{t, t+r}(\Z)$ with $m:=\mbox{lcm}(|g_i|)_{1\leq i\leq r}$), we obtain a basis $B$ of $L$ af follows:
\[B:=(C^T)^{-1}\]

\subsection{Finding a short vector in $L$}

To be safe, one would have to use a SVP algorithm in norm $\ell_\infty$.  The best known algorithm to solve this problem is due to \cite{Aggarwal2018} and its space and time complexities are $2^{0.62t+o(t)}$ and $2^{0.415t+o(t)}$ respectively.  This prohibits any practical implementation. 

However, in practice, one can find very short vectors with polynomial time algorithms. Even LLL is satisfying, despite its exponentially big theoretical bounds. In practice, BKZ with small block size outputs a vector $e\in L$ with $\|e\|_\infty\leq 2r$. 

\subsection{Practical implementation}

With the parameters of \cite[p. 28]{OSIDH}: $\ell=2$, $r=5$, $t=74$ and $n=256$ $K=\Q(i)$,  and $q_1,\cdots, q_t$ the $t$ smallest splitting primes in $\mO_K$ the relation lattice $L$ was computed in 2 h 31 using \verb?SageMath? \cite{sagemath}. This could have been reduced to 1 h 04 if we had expoited the fact that $\mbox{Cl}(\mO_n)$ is cyclic and generated by $[\mf{q}_1]$. The BKZ algorithm was applied to $L$ using the \verb?fpylll? library \cite{fpylll} with a block size $k=4$ to find a vector $e\in L$ of infinity norm $\|e\|_\infty=9<2r$ in less that 0.5 s.

\newpage

\bibliography{bib_report_OSIDH}
\bibliographystyle{unsrt}

\end{document}